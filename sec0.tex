
\subsection{使用する公式の復習}

昇降演算子 (ladder operator) $\hat{a},\hat{a}^\dagger$のリー代数は次の式。
\begin{equation}
  \label{eq:a_algebra}
  [\hat{a},\hat{a}^\dagger] = 1
  \tag{8.15}
\end{equation}
% textlint-disable
エルミート演算子
\begin{equation}
  \hat{N}\equiv\hat{a}^\dagger\hat{a} \tag{8.7}
\end{equation}
と昇降演算子$\hat{a},\hat{a}^\dagger$のリー代数は次の式。
% textlint-enable
\begin{equation}
  \begin{split}
    &[\hat{N},\hat{a}^\dagger]
    =\hat{a}^\dagger[\hat{a},\hat{a}^\dagger]
    =\hat{a}^\dagger, \\
    &[\hat{N},\hat{a}]
    =[\hat{a}^\dagger\hat{a},\hat{a}]
    =[\hat{a}^\dagger,\hat{a}]\hat{a}
    =-\hat{a}
  \end{split}
\end{equation}
このリー代数に対する$N-1$を最高ウェイトとする既約表現をもつ。
これを最高ウェイト表現と呼ぶ。
線形空間$\ab\{\sum_{i=0}^{N-1}k_i\ket{i}|k_i\in\mathbb{C}\}$に対する表現は次の式。
\begin{align}
   & \hat{a}\ket{N-1} = 0                       \\
   & \hat{a}^\dagger\ket{n}=\sqrt{n+1}\ket{n+1}
  \quad(n=0,1,\ldots,N-2)
  \tag{8.2}                                     \\
   & \hat{a}\ket{n}=\sqrt{n}\ket{n-1}
  \quad(n=1,\ldots,N-1)
  \tag{8.5}                                     \\
   & \hat{N}\ket{n}=n\ket{n}
  \tag{8.8}
  \label{eq:number_op_repr}
\end{align}
この表現から線形空間の基底が次の式で書ける。
\begin{equation}
  \label{eq:num_eigenvec}
  \ket{n}
  =\frac{1}{\sqrt{n!}}(\hat{a}^\dagger)^n\ket{0}
  \tag{8.4}
\end{equation}

位置演算子$\hat{x}$、運動量演算子$\hat{p}$の交換関係は以下。
\begin{equation}
  \label{eq:xp_alg}
  [\hat{x},\hat{p}]=i\hbar
  \tag{8.18}
\end{equation}
系を特徴づける長さ$L$を用いると
位置演算子$\hat{x}$、運動量演算子$\hat{p}$から
昇降演算子$\hat{a},\hat{a}$のリー代数をもつ演算子が作れる。
\begin{equation}
  \label{eq:xp2a}
  \hat{a}=\frac{1}{\sqrt{2}}
  \ab(\frac{\hat{x}}{L}+i\frac{L\hat{p}}{\hbar}),\quad
  \hat{a}^\dagger=\frac{1}{\sqrt{2}}
  \ab(\frac{\hat{x}}{L}-i\frac{L\hat{p}}{\hbar})
  \tag{8.17}
\end{equation}
ただし昇降演算子$\hat{a},\hat{a}^\dagger$は無次元量。
% textlint-disable
これは換算プランク定数あるいは Dirac 定数が
\begin{equation}
  \begin{split}
    \hbar\equiv\frac{h}{2\pi}
    &= 1.054571817\ldots\times10^{-34}\si{J.s} \\
    &= 6.582119569\ldots\times10^{-16}\si{eV.s}
  \end{split}
\end{equation}
と定義され、運動エネルギー$K$の公式に対する次元解析が
\begin{equation}
  K=\frac{1}{2}mv^2
  \Rightarrow[K]=ML^2T^{-2}
\end{equation}
となることから
\begin{equation}
  \ab[\frac{Lp}{\hbar}]
  =\frac{[L][p]}{[\hbar]}
  =\frac{L\cdot MLT^{-1}}{ML^2T^{-2}\cdot T}
  = 1
\end{equation}
となって無次元になることが確かめられる。
% textlint-enable
逆に昇降演算子$\hat{a},\hat{a}^\dagger$から
位置・運動量の単位をもつ次の演算子が定義できる。
\begin{equation}
  \label{eq:a2xp}
  \hat{x}=\frac{L}{\sqrt{2}}(\hat{a}+\hat{a}^\dagger),\quad
  \hat{p}=\frac{\hbar}{\sqrt{2}iL}(\hat{a}-\hat{a}^\dagger)
  \tag{8.16}
\end{equation}

孤立系$S$のハミルトニアン$\hat{H}_S$から定まる
状態$\ket{\psi(t)}_S$の時間発展は
次のシュレディンガー方程式で記述される。
\begin{equation}
  \label{eq:schrodinger_eq}
  i\hbar\odv{}{t}\ket{\psi(t)}_S
  =\hat{H}_S(t)\ket{\psi(t)}_S
  \tag{6.18}
\end{equation}
このように状態を時間依存させる記法を
シュレディンガー描像と呼ぶ。

孤立系$S$のハミルトニアン$\hat{H}_S$から定まる
ハイゼンベルグ演算子$\hat{O}_H(t)$の時間発展は
次のハイゼンベルグ方程式で記述される。
\begin{equation}
  \label{eq:heisenberg_eq}
  \odv{}{t}\hat{O}_H(t)
  =\frac{1}{i\hbar}\ab[\hat{O}_H(t),\hat{H}_H(t)]
  \tag{6.36}
\end{equation}
このように演算子を時間依存させる記法を
ハイゼンベルク描像と呼ぶ。

運動量演算子の位置表示はデルタ関数の微分。
\begin{equation}
  \label{eq:p_in_x}
  \braket{x|\hat{p}|x'}
  =-i\hbar\pdv{}{x}\delta(x-x')
  \tag{8.41}
\end{equation}
波動関数$\psi(x)=\braket{x|\psi}$に作用する位置表示の
運動量演算子と位置演算子は次で定義。
\begin{equation}
  \vb{p}=-i\hbar\pdv{}{x},\quad
  \vb{x}=x
  \tag{8.43}
\end{equation}
これは次の意味。
\begin{equation}
  \begin{split}
    &\hat{p}\ket{p}=p\ket{p}
    \Rightarrow
    \braket{x|\hat{p}|p}
    =p\braket{x|p} \\
    &\Leftrightarrow
    p\braket{x|p}
    =\braket{x|\hat{p}|p}
    =\int_{-\infty}^\infty\d x'
    \braket{x|\hat{p}|x'}\braket{x'|p}
    =\int_{-\infty}^\infty\d x'
    \ab(-i\hbar\pdv{}{x}\delta(x-x'))\braket{x'|p} \\
    &\quad\,=
    -i\hbar\pdv{}{x}
    \int_{-\infty}^\infty\d x'
    \delta(x-x')\braket{x'|p}
    =-i\hbar\pdv{}{x}\braket{x|p}
    =\vb{p}\braket{x|p}
  \end{split}
\end{equation}