\subsection{伝搬関数}

系の時間依存性を解析するために波動関数$\psi(t,x)$を
次のように相互相関 (cross-correlation) あるいは畳み込み (convolution) で定義する。
\begin{equation}
  \psi(t,x)
  =\int_{-\infty}^\infty\d x'
  K(t;x,x')\psi(x')
  \tag{9.13}
\end{equation}
ここで$K(t;x,x')$は伝搬関数 (propagator) と呼ばれ、
画像処理の畳み込み処理におけるカーネルと同様の働きをする。
時刻$t=0$の波動関数$\psi(x')$を入力として、
伝搬関数$K(t;x,x')$を用いた畳み込みで時刻$t$の波動関数$\psi(t,x)$を得る。

これより伝搬関数$K(t;x,x')$の発展方程式を明らかにすれば、
任意の波動関数$\psi(x')$の時間発展も求まる。
(\ref{eq:harmonic_oscillator_H_in_x})式から次を得る。
\begin{equation}
  \begin{split}
    0&=\ab(i\hbar\pdv{}{t}-\vb{H})\psi(t,x)
    =\ab(i\hbar\pdv{}{t}-\vb{H})
    \int_{-\infty}^\infty\d x' K(t;x,x')\psi(x') \\
    &=\int_{-\infty}^\infty\d x'
    \ab[\ab(i\hbar\pdv{}{t}-\vb{H})K(t;x,x')]\psi(x') \\
  \end{split}
\end{equation}
ここで上記が任意の$\psi(x')$で成り立つためには次が必要。
\begin{equation}
  \label{eq:propagator_evolution}
  \begin{split}
    i\hbar\pdv{}{t}K(t;x,x')
    =\vb{H}K(t;x,x')
    =\ab(-\frac{\hbar^2}{2m}\pdv[order={2}]{}{x}+\frac{m\omega^2}{2}x^2)K(t;x,x')
  \end{split}
  \tag{9.12}
\end{equation}
この偏微分方程式は$\psi(t=0,x)=\psi(x)$を初期条件として解く。
この初期条件は次の条件と等価である。
\begin{equation}
  \psi(x)=\psi(t=0,x)
  =\int_{-\infty}^\infty\d x'
  K(t=0;x,x')\psi(x')
  \Leftrightarrow
  K(t=0;x,x')=\delta(x-x')
\end{equation}
この微分方程式の形式的な解は次の式である。
\begin{equation}
  K(t;x,x')=\braket{x|\exp \ab(-\frac{it}{\hbar}\hat{H})|x'}
\end{equation}
これは$\braket{x|\hat{H}|x'}=\vb{H}\braket{x|x'}$を用いて次で確かめられる。
\begin{equation}
  \begin{split}
    &i\hbar\pdv{}{t}K(t;x,x')
    =i\hbar\pdv{}{t}\braket{x|\exp \ab(-\frac{it}{\hbar}\hat{H})|x'}
    =\braket{x|\ab[i\hbar\pdv{}{t}\exp \ab(-\frac{it}{\hbar}\hat{H})]|x'} \\
    &=\braket{x|\hat{H}\exp \ab(-\frac{it}{\hbar}\hat{H})|x'}
    =\int\d x''\braket{x|\hat{H}|x''}\braket{x''|\exp \ab(-\frac{it}{\hbar}\hat{H})|x'} \\
    &=\int\d x''\vb{H}\delta(x-x'')K(t;x'',x')
    =\vb{H}K(t;x,x')
  \end{split}
\end{equation}
一方で (\ref{eq:harmonic_oscillator_schrodinger}) 式を
伝搬関数$K(t;x,x')$を用いて解く代わりにグリーン関数法で解く場合は、
次の偏微分方程式の解であるグリーン関数$G(t;x,x')$を求める。
\begin{equation}
  \label{eq:green_function_eq}
  \ab(i\hbar\pdv{}{t}-\vb{H})G(t;x,x')=\delta(t)\delta(x-x')
\end{equation}
これはヘヴィサイドの階段関数$\Theta(t)$を用いて次のように定義することで満たせる。
\begin{equation}
  G(t;x,x')=\frac{1}{i\hbar}\Theta(t)K(t;x,x')
\end{equation}
これは (\ref{eq:propagator_evolution}) 式と
$\partial_t\Theta(t)=\delta(t)$となることを利用すると次のように確認できる。
\begin{equation}
  \begin{split}
    &\ab(i\hbar\pdv{}{t}-\vb{H})\ab(\frac{1}{i\hbar}\Theta(t)K(t;x,x'))
    =\delta(t)K(t;x,x')+\Theta(t)\ab(i\hbar\pdv{}{t}-\vb{H})K(t;x,x') \\
    &=\delta(t)K(0;x,x')=\delta(t)\delta(x-x')
  \end{split}
\end{equation}
この$G(t;x,x')$の構成は遅延グリーン関数と呼ばれる。

(\ref{eq:green_function_eq}) 式を満たすグリーン関数は
遅延グリーン関数の他にも先進グリーン関数がある。
どちらを選択するかはグリーン関数$G(t;x,x')$の複素構造の扱い方で決まる。
これを確認するために自由粒子$\omega=0$に対するグリーン関数$G(t;x,x')$を調べる。
複素構造を確認するために次のフーリエ変換を用いる。
\begin{equation}
  \begin{split}
    G\equiv
    \int \d\omega\d x
    \tilde{G}(\omega,k)e^{-i\omega t+ikx},\quad
    \delta(t)=\frac{1}{2\pi}\int\d\omega e^{-i\omega t},\quad
    \delta(x-x')=\frac{1}{2\pi}\int\d k e^{ik(x-x')}
  \end{split}
\end{equation}
これを$\omega=0$のときの (\ref{eq:green_function_eq}) 式に代入して次を得る。
\begin{equation}
  \begin{split}
    &\ab(i\hbar\pdv{}{t}+\frac{\hbar^2}{2m}\pdv[order={2}]{}{x})
    \int \d\omega\d k
    \tilde{G}(\omega,k)e^{-i\omega t+ikx}
    =\frac{1}{4\pi^2}
    \int\d\omega\d k e^{-i\omega t+ik(x-x')} \\
    &\Leftrightarrow
    \int\d\omega\d k
    \ab[\ab(\hbar\omega-\frac{(\hbar k)^2}{2m})\tilde{G}(\omega,k)-\frac{1}{4\pi^2}e^{-ikx'}]
    e^{-i\omega t+ikx} = 0
  \end{split}
\end{equation}
ここで任意の$t,x$で成立する必要があるので次を得る。
\begin{equation}
  \begin{split}
    \ab(\hbar\omega-\frac{(\hbar k)^2}{2m})\tilde{G}(\omega,k)
    =\frac{1}{4\pi^2}e^{-ikx'}
    \Leftrightarrow
    \tilde{G}(\omega,k)
    =\frac{1}{4\pi^2}\frac{e^{-ikx'}}{\hbar\omega-\frac{(\hbar k)^2}{2m}}
  \end{split}
\end{equation}
これよりフーリエ変換したグリーン関数$\tilde{G}(\omega,k)$は
$\omega\equiv\omega_0=\frac{\hbar k^2}{2m}$に孤立特異点をもつ。
この孤立特異点を避けて主値積分をする必要がある。
主値積分を評価するの際の積分経路の取り方の違いで
遅延・先進グリーン関数のどちらを選択するか決定される。

対象の主値積分を次で定義する。
\begin{equation}
  \begin{split}
    &I\equiv I_1+I_2 \\
    &I_1\equiv
    \lim_{R\rightarrow\infty}
    \lim_{\epsilon\rightarrow0}
    \int_{-R}^{\omega_0-\epsilon}\d\omega
    \frac{1}{4\pi^2}\frac{e^{-ikx'}}{\hbar\omega-\frac{(\hbar k)^2}{2m}}
    e^{-i\omega t} \\
    &I_2\equiv
    \lim_{R\rightarrow\infty}
    \lim_{\epsilon\rightarrow0}
    \int_{\omega_0+\epsilon}^R\d\omega
    \frac{1}{4\pi^2}\frac{e^{-ikx'}}{\hbar\omega-\frac{(\hbar k)^2}{2m}}
    e^{-i\omega t}
  \end{split}
\end{equation}
$R$によるカットオフは後程の議論のために導入した。
また$k$に関するフーリエ変換は後ほど評価する。

% textlint-disable
\begin{figure}[tbp]
  \begin{center}
    \begin{tikzpicture}[%詳細設定(気にしない)
        arrow/.style={
            postaction={
                decorate,
                decoration={
                    markings,
                    mark=at position #1 with {\arrow{stealth}}
                  }
              }
          },label2/.style 2 args={
            pos/.style={
                postaction={
                    decorate,
                    decoration={
                        markings,
                        mark=at position ##1 with \node #2;
                      }
                  }
              }
          },label/.style={
            label2={1}{#1}
          },pos/.default=.5
        ,arrow/.default=.5
      ]

      % 複素数平面の描画
      \draw[-{Stealth}] (-4.5,0) -- (4.5,0) node[right]{$\Re \omega$};
      \draw[-{Stealth}] (0,-0.5) -- (0,4) node[above]{$\Im \omega$};
      % 原点の描画
      \draw (0,0) node[below left]{O};

      % 座標の定義
      \coordinate (min_Rw) at (-3, 0);
      \coordinate (left_cutoff) at (0.9, 0);
      \coordinate (right_cutoff) at (2.1, 0);
      \coordinate (max_Rw) at (3, 0);
      \coordinate (top) at (0, 3);
      \coordinate (singular) at (1.5, 0);
      % 座標の描画
      \draw (min_Rw) node[below] {$\omega=-R$};
      \draw (max_Rw) node[below right] {$\omega=R$};
      \draw (top) node[above right] {$\omega=Ri$};
      \draw (singular) node[below=0.2cm] {$\omega=\omega_0$};

      % 特異点
      \draw[
        only marks,
        mark=x,
        mark size=4pt
      ] plot (singular);

      % 直線経路
      \draw[
        very thick,
        arrow
      ] (min_Rw) -- (left_cutoff) node[midway,above] {$I_1$};
      \draw[
        very thick,
        arrow=.5
      ]  (right_cutoff) -- (max_Rw) node[midway,above] {$I_2$};

      % 円弧経路
      \draw[
        very thick,
        domain=0:180,
        variable=\t,
        label={[above]{$\Gamma_R$}},
        pos={.25},
        arrow=.75
      ] plot ({3*cos(\t)},{3*sin(\t)});%pos,arrowの両方ともに省略していない
      \draw[
        very thick,
        domain=180:0,
        variable=\t,
        label={[above]{$\Gamma_\epsilon$}},
        pos={.25},
        arrow=.5
      ]  plot ({1.5+0.6*cos(\t)},{0.6*sin(\t)});%両方ともに省略していない

    \end{tikzpicture}
    \caption{遅延グリーン関数の積分経路}\label{fig:retarded_green_function}
  \end{center}
\end{figure}
% textlint-enable
